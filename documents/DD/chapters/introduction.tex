\section{Purpose}
The Design Document aims to give usefull information to help in software development by providing the details for how the software should be built. In particolar it should be detailed enough so that developers could code the project without having to make any significant decisions. \\
This is done thanks to detailed description with graphical documentation of the software design for the project including different diagram types and other supporting requirement informations.
\section{Scope}
The main scope of the system is to provide users the possibility to make a booking in order to give access to the market. This could be done with two options: the first allows users to be inserted in the virtual queue; instead, the second give the possibility to schedule the booking in a precise moment in an particular day. \\
So, the system have to reply users' requests in real time without waiting more than few seconds due to its reliability. \par
To achieve this, the system is organized with a \textbf{3 tiers architecture} which divides the systems in independent modules: presentation, application and a data tier. The detailed architecture will be described as well in the next chapter.

\end{comment}
\bigskip
\bigskip
\section{Definitions, Acronyms, Abbreviations}
See also definitions already menitoned in the RASD.
\subsection{Definitions}
\begin{itemize}
\item \textbf{CLup} mobile applicaiton used by users;
\item \textbf{CLup Operator}: desktop applicaiton used by Receptionist;
\item \textbf{Application Server}: part of the application layer. It refers to the server needed to run the 
\item \textbf{Booking}: it's the generic appointment. It could be either a Visit or a Reservation;
\item \textbf{Avarage shopping time}: it corresponds to mean considering all appointments of all the users of the Market; 
\item \textbf{Bottom navigation bar}: graphical object that allows the user to display different destinations at the bottom of a screen;
\item \textbf{Horizontal fragmentation}: it consist in divide a table into a set of smaller table;
\item \textbf{Cross-platform development}: software development that consists to build an application using a universal language;
\item \textbf{Servelet}: program that runs within a Web server;
\item \textbf{SMS Gateway}: service that allows a computer to send or receive text; messages; 
\end{itemize}
\subsection{Acronyms}
\begin{itemize}
\item \textbf{DB}: Database;
\item \textbf{ACID}: Atomicity, Consistency, Integrity and Durability;
\item \textbf{DBMS}: Database Management System;
\item \textbf{RDBMS}: Relational Database Management System;
\item \textbf{SQL}: Structured Query Language;
\item \textbf{HTTPS}: Hypertext Transfer Protocol Secure;
\item \textbf{MVC}: Model View Controller;
\item \textbf{SMS}: Short Message Service;
\item \textbf{CI}: Continous Integration;
\item \textbf{CD}: Continous Delivery;
\item \textbf{QA}: Quality Assurance;
\end{itemize}


\subsection{Abbreviations}
\begin{itemize}
\item \textbf{Rn}: n-th requirement;
\end{itemize}


\section{Revision history}
\section{Reference Documents}
This document is strictly based on:
\begin{itemize}
\item The specification of the \textbf{RASD and DD assignment} of the Software Engineering II corse, held by professor Matteo Rossi and Elisabetta Di Nitto at the Politecnico di Milano, A.Y 2020/2021;
\item \textbf{Slides} of Software Engineering 2 course on BEEP;
\end{itemize}
\section{Document Structure}
\begin{itemize}
\item[1]\textbf{Introduction}: it gives an overview of the document, by providing a brief introduction of the problem and information about the terminology used;
\item[2]\textbf{Architectural Desing}: this section aims to describe mainly the system’s structure. In particular it's focused on its components the interaction between them and all necessary details for a possible implementation; 
\item[4]\textbf{User Interface Design}: it provides the user interfaces of our system in order to understand the flow of the GUI part of our system, in relation of the actions done. In particular is focused on the UI of CLup and CLup Operator; 
\item[4]\textbf{Requirements Traceability}: in this section we link each requirement already mentioned in the RASD to the main component of our architecture;
\item[5]\textbf{Implementation, Integration and Test Plan}: it's focused on the development process, focusing on how it will be integrated and tested;
\item[6]\textbf{Effort Spent}: it shows the time spent to realize this document, divided for each section;
\item[7]\textbf{References}: it contains the references to any documents and to the Softwares used in this document.
\end{itemize}


