\begin{figure}[H]
  \caption{Three tiers architecture of the system}
  \label{3tiers}
  \centering
  \includegraphics[scale=0.25]{diagrams/3_tiers.jpeg}

\end{figure}


\section{Overview: High-level components and their interaction}
The system is organized following the three tiers architecture. This aims to decouple logical layers in order to gurantee an horizontal scalability and a low fault tolerance.
Graphically it's shown in the figure~\ref{3tiers}.
\par


\textbf{Presentation layer}. It's the front end layer which consists of the user interface. We have two types of user interface, depending on his functionality: 
\begin{itemize}
\item \textbf{CLup}: It's the mobile application used by users who have a smartphone. They can manage their booking by themselves;
\item \textbf{CLup Operator}: It's the desktop application used by receptionists that act as an intermediary to manage booking of users that have only a mobilephone.
\end{itemize}

\textbf{Application layer}. It deals with the model of the system, by containing the business logic of the application. In our system it consists in a remote server to which mobile and desktop applications have to connect due to manage any bookings.

\textbf{Data layer}. It's composed by a data storage system. It includes: 

\begin{itemize}
\item User sensitive data asked during the registration process;
\item Information about user's grocery shopping;
\end{itemize}



\section{Component view}


\begin{figure}[H]
  \caption{High-Level component diagram}
  \label{highlevel}
  \centering
  \includegraphics[scale=0.25]{diagrams/h_level.png}

\end{figure}
component diagram ogni componente descritto
er diagram o class diagram specificp

-strittura
-model applicazione
-database: dbms proprietà acid


In our architecture the data layer is composed by a relational DB needed to store informations about users and their dynamics in the market. In particular it should be connected to the server placed in the application layer. In order to reach the goal we plan to adopt a RDBMS in order to deal with a relational database. In addition, it ensures 
consistency of data due to the ACID properties.
Morover, it includes a software program which is designed to capture request over a network of the application server. From this user and receptionist in fact could retrieve informations needed through SQL Query by connecting to it.  
An other important aspect is the security of the data due to mitigate any risks of violation. In order to do that we limit its access exclusively only to the application server. Communication between them will be also encrypted and accounts' password will be hashed.




\section{Deployment view}
-deployment diagram

\section{Runtime view}
sequence diagrams

\section{Component interfaces}
ogni componente
app
server+db
laptop receptionist

\section{Selected architectural styles and patterns}
mvc + tier + ..

\section{Other design decisions}

security+google api
asyncrnous coomunication because eccc
